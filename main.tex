\documentclass{book}
\usepackage{titlesec}
\usepackage{xeCJK}
\usepackage{graphicx}
\usepackage{wrapfig}
\usepackage{fancyhdr}
\usepackage{tocstyle}
\usepackage{setspace}
\usepackage[top=3cm,bottom=4cm,left=3.2cm,right=2.5cm]{geometry}
\pagestyle{fancy}

\fancyhf{} % clear all headers and footers
\renewcommand{\headrulewidth}{0pt} % remove rule between header and text
\fancyhead[LE,RO]{\thepage} % put page number in left header on even pages,
                            % right header on odd pages
\fancyhead[RE]{\nouppercase{\leftmark}} % remove uppercase on chapter title
\renewcommand{\chaptermark}[1]{\markboth{#1}{}} % remove "Chapter N." prefix


\renewcommand*{\contentsname}{目录}


\setCJKmainfont{MS Song.ttf}

\newcommand{\degrees}{\ensuremath{^\circ}}

\titleformat{\part}[display]{\Huge}{第 \thepart 部}{0pt}{\Huge\bfseries}
\titleformat{\chapter}[display]{\Huge}{第 \thechapter 章}{0pt}{\Huge\bfseries}

\begin{document}
\Large

\begin{spacing}{1.4}
\tableofcontents
\end{spacing}

\setstretch{1.4}
%\linespread{1.4}


\part{这个国家}

\chapter{二维国的自然状况}
\input{ch1.txt}

\chapter{二维国的风土人情}
\input{ch2.txt}

\chapter{关于二维国的国民}
\input{ch3.txt}

\chapter{关于妇女们}
\input{ch4.txt}

\chapter{我们相互辨认的方法}
\input{ch5.txt}

\chapter{视觉辨认}
\input{ch6.txt}

\chapter{不规则图形}
\input{ch7.txt}

\chapter{历史上的色彩热}
\input{ch8.txt}

\chapter{《着色议案》}
\input{ch9.txt}

\chapter{镇压着色革命}
\input{ch10.txt}

\chapter{神职人员}
\input{ch11.txt}

\chapter{神职人员的教义}
\input{ch12.txt}

\part{其他国家}

\chapter{我对一维国的访问}
\input{ch13.txt}

\chapter{问国王解释二维国的一留徒劳}
\input{ch14.txt}

\chapter{来自三维国的陌主人}
\input{ch15.txt}

\chapter{陌生人间我揭示三维国奥秘的一番徒劳}
\input{ch16.txt}

\chapter{徒劳一场的球又求助于行动}
\input{ch17.txt}

\chapter{我如何来到三维国,以及在这里的所见所闻}
\input{ch18.txt}

\chapter{球向我展示了三维国的其它秘密,我却仍然不满足;事情的最后结局}
\input{ch19.txt}

\chapter{我在梦中受到球的鼓励}
\input{ch20.txt}

\chapter{我试图向孙子传授三维理论,其过程及结果}
\input{ch21.txt}

\chapter{我试图以其它方式散播三维理论,结局如何}
\input{ch22.txt}

\end{document}